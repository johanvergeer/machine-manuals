\refstepcounter{chapter}
\addcontentsline{toc}{chapter}{Work Tables}

\section{Fixed Angle Table}

\begin{figure}[h]
    \centering
    % \includegraphics[width=0.8\textwidth]{fixed_angle_table.jpg}
\end{figure}

\subsection*{Application}
The fixed angle table has a rectangular shape and is used to hold bulky workpieces during machining operations that do not require an angular adjustment of the workpiece.

The fixed angle table is fastened to the machine’s vertical clamping table using T-slot nuts and hex bolts.

\subsection*{Technical Data}
\begin{tabbing}
\hspace{4cm} \= \hspace{4cm} \= \kill
\textbf{Clamping surface} \> mm \> 800 x 250 \\
\textbf{Number of T-slots (14 H7)} \> \> 4 \\
\textbf{Distance between T-slots} \> mm \> 63 \\
\textbf{Weight (approx.)} \> kg \> 100 \\
\textbf{Maximum table load (incl. workpiece and clamping tools)} \> kg \> 200
\end{tabbing}

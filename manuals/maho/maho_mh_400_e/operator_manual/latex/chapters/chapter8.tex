\refstepcounter{chapter}
\addcontentsline{toc}{chapter}{Spare Parts Plans and Lists}

\setsectiontitle{Guidelines for Ordering Spare Parts}

\setcounter{section}{0}

\subsection*{8.1 Wear and Spare Parts List}

The following pages list wear and spare parts assigned to the main assemblies.  
They are categorized using various abbreviations, which are listed under the  
\enquote{Material/Remarks} column.

The abbreviations mean:

\begin{table}[H]
    \centering
    \renewcommand{\arraystretch}{1.2}
    \begin{tabular}{|p{0.25\textwidth}|p{0.7\textwidth}|}
        \hline
        \textbf{Abbreviation} & \textbf{Meaning} \\
        \hline
        VERSCH & Wear part (no warranty claim) \\
        \hline
        ET-1 & Spare part (normal wear) \\
        \hline
        ET-2 & Spare part (maximum equipment package) \\
        \hline
    \end{tabular}
\end{table}

\subsection*{8.2 Spare Parts Catalog}

\textbf{Mechanical spare parts} are compiled in a separate spare parts  
documentation, which is provided once per machine.

This documentation includes:

\begin{itemize}
    \item \textbf{Assembly drawings}
    \item \textbf{Parts lists}, including part designation, quantity, and order/ID number.
\end{itemize}

\subsection*{8.3 Spare Parts for Electrical Equipment}

\begin{itemize}
    \item All electrical components are represented in the machine’s wiring diagram  
          (device arrangement of the machine) and assigned a function group number.
    \item The device list of the wiring diagram contains, for each function group number,  
          the \\designation and identification number of the spare part.
    \item To ensure fast and correct delivery of spare parts, the order must include:
    \begin{itemize}
        \item \textbf{Machine type}
        \item \textbf{Factory number} of the machine.
        \item \textbf{ID number and designation} of the spare part (for electrical parts,  
              also the \enquote{function group number}) or the page number from the  
              operator’s manual and position number of the part on that page.
        \item \textbf{Color} (only for painted parts).
    \end{itemize}
\end{itemize}

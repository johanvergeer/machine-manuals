\documentclass[openany,11pt]{book}

\usepackage[a4paper,left=20mm,right=20mm,top=20mm,bottom=5mm,twoside=false]{geometry}
\usepackage{fancyhdr}
\usepackage{graphicx}
\usepackage{background} % <-- NEW: Add background package
\usepackage{tikz}
\usepackage{tcolorbox} 
\usepackage{titlesec}     % Allows customization of section/chapter titles
\usepackage{placeins} % Add at the top for FloatBarrier

\renewcommand{\familydefault}{\ttdefault}

\pagestyle{fancy}
\fancyhf{} % Clear default header/footer

% Adjust header height
\setlength{\headheight}{50pt}  % Increase height to fit text
\setlength{\headsep}{15mm}      % Push content down

% Define header
\fancyhead[L]{%
    \textbf{MAHO}\\
    Corporation\\
    D-8962 Pfronten
}
\fancyhead[C]{%
    \raisebox{-8mm}{%
        OVERVIEW OF ERROR CODES - CNC 432
    }
}

\newcounter{basepagenum}  % Define a new counter
\setcounter{basepagenum}{26778}  % Set base page number
\fancyhead[R]{%
    \textbf{E4.\the\numexpr\value{basepagenum}+\thepage-1\relax}\\[3mm] % Dynamic page number
    Date Created: 09/06/87\\
    Last Revision Date: 30/01/90\\
    Drawn By: HRU/LOH
}

\renewcommand{\headrulewidth}{0pt} % Remove default head rule

\backgroundsetup{%
  scale=1,
  color=black,
  opacity=1,
  angle=0,
  position=current page.north west,
  vshift=-3cm,
  hshift=2cm,
  nodeanchor=north west,
  contents={%
    \begin{tikzpicture}[remember picture, overlay]
      \node[anchor=north west, inner sep=0] (box) at (0, 0) {
        \begin{tcolorbox}[
          colframe=black, 
          colback=white, 
          boxrule=0.3mm,
          leftrule=0mm,
          rightrule=0mm,
          sharp corners, 
          width=\dimexpr\paperwidth-3cm\relax,
          height=1cm
        ]
        \end{tcolorbox}
      };
      \draw[black, thick] ([yshift=-24cm]box.south west) -- ([yshift=-24cm]box.south east);
      \node[anchor=south, yshift=-26cm] at ([yshift=1cm]box.south) {
        \footnotesize \textit{Unofficial translation - Not affiliated with DMG-Mori or Maho.}
      };
    \end{tikzpicture}
  }
}

% Section titles: Bold, number in margin
\titleformat{\section}
  [hang]
  {}
  {\thesection}
  {0pt}
  {\underline}
\titlespacing{\section}{0pt}{-10pt}{-10pt}

% \renewcommand{\arraystretch}{0.85} % Adjust this value as needed

\begin{document}

\begin{table}[h]
    \begin{tabular}{ll}
        \textasteriskcentered01 & System program error \\
        \textasteriskcentered02 & Calculation error \\
        \textasteriskcentered03 & NC temperature exceeds 65°C \\
        \textasteriskcentered05 & Emergency stop triggered by machine \\
        \textasteriskcentered95 & System error (calculation error) \\
        \textasteriskcentered96 & System error (division by zero) \\
        \textasteriskcentered97 & System error (overflow) \\
        \textasteriskcentered98 & System error (unexpected interrupt) \\
        \textasteriskcentered99 & System error (debug mode) \\
    \end{tabular}
\end{table}

\vspace{-5mm}
\noindent Note: Instead of ‘\textasteriskcentered’, the hash symbol (\#) is also used.

\begin{table}[h]
    \begin{tabular}{lll}
    M01 & Checksum Error & Machine constants memory (RAM) \\
    M02 & Checksum Error & Tool memory (RAM) \\
    M03 & Checksum Error & Background memory for machine constants (RAM) \\
    M20 & Checksum Error & EPROM 1-8 or system PROM checksum error \\
    M21 & Checksum Error & EPROM 9-12 \\
    M22 & Checksum Error & EPROM 13-16 \\
    M23 & Checksum Error & EPROM 23, 24, or 25 \\
    M60 & Checksum Error & Workpiece program/macro program memory (RAM) \\
    M70 & Communication Error & Hardware fault in communication processor \\
    \end{tabular}
\end{table}

\section*{Axes}

\begin{table}[h]
    \begin{tabular}{lll}
    (X,Y,Z,A,B,C) & 01 & Pre-alarm: Linear measurement system \\
    (X,Y,Z,A,B,C) & 02 & Alarm: Measurement system error \\
    (X,Y,Z,A,B,C) & 03 & Power supply failure in the measurement system \\
    (X,Y,Z,A,B,C) & 04 & Maximum following error exceeded \\
    (X,Y,Z,A,B,C) & 05 & Software limit switch triggered \\
    (X,Y,Z,A,B,C) & 06 & Calculated axis speed too high, \\
                     &    & based on rapid traverse speed in NC \\
    (X,Y,Z,A,B,C) & 07 & Standstill monitoring activated \\
    (X,Y,Z,A,B,C) & 08 & Calculated axis speed too high, \\
                     &    & based on feed rate in NC \\
    (X,Y,Z,A,B,C) & 09 & Dynamic following error exceeded \\
    (X,Y,Z,A,B,C) & 10 & Linear compensation: Correction values sum > 200 increments \\
    (X,Y,Z,A,B,C) & 11 & Linear compensation: Support point distance < 100 increments \\
    (X,Y,Z,A,B,C) & 12 & Cyclic compensation: Correction values sum > 200 increments \\
    (X,Y,Z,A,B,C) & 13 & Cyclic compensation: Support point distance < 100 increments \\
    \end{tabular}
\end{table}

\FloatBarrier 

\section*{Main Spindle Drive}

\begin{table}[!h]
    \begin{tabular}{ll}
    S01 & Pre-alarm: Measurement system \\
    S02 & Alarm: Measurement system error \\
    S03 & Power supply failure in the measurement system \\
    S04 & Maximum following error exceeded \\
    S05 & Position window not reached \\
    S07 & Standstill monitoring activated \\
    \end{tabular}
\end{table}

\section*{Electronic Handwheel}

\begin{table}[!h]
    \begin{tabular}{ll}
    W1 (101)  & Pre-alarm: Measurement system \\
    W2 (102)  & Alarm: Measurement system error \\
    W3 (103)  & Power supply failure in the measurement system \\
    W105      & Software limit switch triggered \\
    \end{tabular}
\end{table}

\newpage

\end{document}